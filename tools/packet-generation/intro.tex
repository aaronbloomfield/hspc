\section*{\thelongtitle}

Welcome to the \thelongtitle. Before you
start the contest, please be aware of the following notes:

\section*{Rules}

\begin{enumerate}

\item \introbullet

\problemtable

\item Solutions for problems submitted for judging are called
  runs. Each run will be judged.

The judges will respond to your submission with one of the following
responses. In the event that more than one response is applicable, the
judges may respond with any of the applicable responses.

\begin{center}
\begin{tabular}{|c|p{4in}|}\hline
\bf Response & \bf Explanation \\ \hline \hline
\bf Yes & Your submission has been judged correct. \\
\bf No - Wrong Answer & Your submission generated output that is not correct or is incomplete. \\
\bf No - Output Format Error & Your submission's output is not in the correct format or is misspelled. \\
\bf No - Excessive Output & Your submission generated output in addition to or instead of what is required. \\
\bf No - Compilation Error & Your submission failed to compile. \\
\bf No - Run-Time Error & Your submission experienced a run-time error. \\
\bf No - Time Limit Exceeded & Your submission did not terminate within one minute. \\ \hline
\end{tabular}
\end{center}

\item A team's score is based on the number of problems they solve and
  penalty minutes, which reflect the amount of time and number of
  incorrect submissions made before the problem is solved. For each
  problem solved correctly, penalty minutes are issued equal to the
  time at which the problem was solved plus 20 minutes for each
  incorrect submission. No penalty minutes are added for problems that
  are never solved. Teams are ranked first by the number of problems
  solved and then by the fewest penalty minutes.

\item This problem set contains sample input and output for each
  problem. However, the judges will test your
  submission against several other more complex datasets, which will
  not be revealed until after the contest. One challenge is
  designing other input sets for yourself so that you may fully test
  your program before submitting your run. Should you receive a
  ``wrong answer" judgment, you should consider what other datasets you
  could design to further evaluate your program.

\item In the event that you think a problem statement is ambiguous or
  incorrect, you may request a clarification. Read the problem
  carefully before requesting a clarification. If the judges believe
  that the problem statement is sufficiently clear, you will receive
  the response, ``The problem statement is sufficient; no clarification
  is necessary." If you receive this response, you should read the
  problem description more carefully. If you still think there is an
  ambiguity, you will have to be more specific or descriptive of the
  ambiguity you have found. If the problem statement is ambiguous in
  specifying the correct output for a particular input, please include
  that input data in the clarification request.

You may not submit clarification requests asking for the correct
output for inputs that you provide. Sample inputs may be useful in
explaining the nature of a perceived ambiguity, e.g., ``There is no
statement about the desired order of outputs. Given the input: \ldots ,
would not both this: \ldots and this: \ldots be valid outputs?".

If a clarification that is issued during the contest applies to all
the teams, it will be broadcast to everybody.

\item Runs for each particular problem will be judged in the order
  they are received. However, it is possible that runs for different
  problems may be judged out of order. For example, you may submit a
  run for B followed by a run for C, but receive the response for C
  first.

{\bf Do not} request clarifications on when a response will be
returned. If you have not received a response for a run within 30
minutes of submitting it, {\bf you may have a runner ask the site
  judge to determine the cause of the delay. Under no circumstances
  should you ever submit a clarification request about a submission
  for which you have not received a judgment.}

If, due to unforeseen circumstances, judging for one or more problems
begins to lag more than 30 minutes behind submissions, a clarification
announcement will be issued to all teams.  This announcement will
include a change to the 30 minute time period that teams are expected
to wait before consulting the site judge.

\item The submission of abusive programs or clarification requests to
  the judges will be considered grounds for immediate
  disqualification.  This includes submitting dozens of runs within a
  short time period (say, within a minute or two).

\end{enumerate}

\section*{Your Programs}

\begin{enumerate}[resume]

\item All solutions must read from standard input and write to
  standard output. In C this is {\tt scanf()} / {\tt printf()}, in C++
  this is {\tt cin} / {\tt cout}, and in Java this is {\tt System.in}
  / {\tt System.out}. The judges will ignore all output sent to
  standard error ({\tt cerr} in C++ or {\tt System.err} in Java). You
  may wish to use standard error to output debugging information. From
  your workstation you may test your program with an input file by
  redirecting input from a file:

\begin{verbatim}
program < file.in
\end{verbatim}

\item All lines of program input and output should end with a newline
  character ({\tt {\verb$\$}n}, {\tt endl}, or {\tt println()}).

\item All input sets used by the judges will follow the input format
  specification found in the problem description. You do not need to
  test for input that violates the input format specified in the
  problem.

\item Unless otherwise specified, all lines of program output should
  be left justified, with no leading blank spaces prior to the first
  non-blank character on that line.

\item Unless otherwise specified, all numbers in your output should
  begin with a '-' if negative, followed immediately by 1 or more
  decimal digits. If it is a real number, then the decimal point
  should be followed by as many decimal digits as can be printed.
  This means that for floating point values, use standard printing
  techniques (cout and System.out.println). Unless otherwise noted,
  the judging will check your programs with $10^{-3}$ accuracy, so
  only consider the sample output up until that point.

In simpler terms, neither scientific notation nor commas will be used
for numbers, and you should ensure you do not round or use a set
precision unless otherwise specified in the problem statement.

\item If a problem specifies that an input is a floating point number,
  the input will be presented according to the rules stipulated above
  for output of real numbers, except that decimal points and the
  following digits may be omitted for numbers with no fractional component.
    Scientific notation will not be used in input sets unless
  a problem statement explicitly specifies it.

\end{enumerate}

Good luck, and HAVE FUN!!!
