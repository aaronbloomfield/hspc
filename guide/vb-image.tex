This appendix lists the full installation directions for the
VirtualBox image that we have used in our contests for the last few
years.  They are based on what can be found
online\footurl{http://aaronbloomfield.github.io/slp/docs/virtualbox-image-details.html}.
The version online has more sections than what is listed here, as the
original image is used at UVa for various classes.  Only the sections
relevant to a programming contest are included here.

This system uses 64 bit Ubuntu 16.04.


\section{Introduction}

\subsection*{Software Versions}

This installation document installs the following versions:

\begin{itemlist}
\item Kubuntu 16.04, 64-bit\footurl{http://kubuntu.org/getkubuntu/}
\item Clang++ version 3.8.0
\item Python 3.5.2\footurl{http://packages.ubuntu.com/xenial/python3}
\end{itemlist}

Newer versions of the above may have since come out, but at the time
of the writing of this document (August 2017), they were either the
versions installed via apt-get (Clang, Python).

\subsection*{Notes}

\begin{itemlist}
\item The guest hard drive reported 9.3 Gb of space available, prior
  to distribution of the image.  The disk image itself was, after
  compaction, 7.6 Gb.  When compressed via zip, it was 2.6 Gb in size.
\item You will likely need to use a different unzip program to extract
  the image; the ones that come bundled with most operating systems
  can not handle archives of that size.  We have used
  7-zip\footurl{http://www.7-zip.org/} with success.
\item Video and sound (via youtube) worked fine with Chrome.
\item The VM may capture the mouse~-- to uncapture it, you press the
  "host key", (which is the right Control key on Linux and Windows
  hosts, and the left Command key on Mac hosts).  To have it warn you
  about what this is, you can reset all warnings via the VirtualBox
  help menu, and it will warn you about this at boot-up.
\item In the image creation process, you may run into a problem with
  VirtualBox where it cannot register a new (or different) disk
  because it has the same UUID as a previous disk that you are
  replacing.  If so, then the command {\tt VBoxManage internalcommands
  sethduuid disk.vdi} (changing {\tt disk.vdi} appropriately) will change
  the UUID, and allow you to proceed.
\end{itemlist}


\section{Basic installation}

\begin{itemize}
\tightlist
\item Created a new VirtualBox image

  \begin{itemize}
  \tightlist
  \item I named it ``UVa HSPC 2017'' or similar; I manually selected
    ``Ubuntu (64 bit)'' as the version
  \item I set the memory at 1536 Mb (instead of the default of 512
    Mb), ensured that the disk size was ``dynamically allocated'' and
    was set to 20 Gb (instead of the default 8 Gb); everything else
    was set at the default.  If your computers can support it, setting
    the memory at 2048 Mb would be better.
  \end{itemize}
\item  I installed the latest Kubuntu 16.04.1 LTS (64 bit), desktop edition, from the
  DVD image online\footurl{http://www.kubuntu.org/getkubuntu}.

  \begin{itemize}
  \tightlist
  \item When prompted, I clicked on `download updates' and `install
    3rd party software' when the options were given
  \item For hard drive, I used the default: ``Guided -- use entire disk''
  \item The computer name is cassiopeia, the login name is `student', full
    name is `L33t H4x0r', and the password is `password'
  \item  This account can run root (system) commands via `sudo' - if you
    don't know what this means, you can safely ignore it
  \end{itemize}
  
\item Once it was finished, I rebooted, and logged in
\item Via a Konsole, ran \texttt{sudo\ apt-get\ update} then
  \texttt{sudo\ apt-get\ dist-upgrade}
\item Reboot!
\item Ran \texttt{apt-get\ autoremove} (which didn't have to remove
  anything)
\item  VirtualBox guest additions

  \begin{itemize}
  \tightlist
  \item These are the utilities so that VirtualBox will work correctly
    with the host computer (proper full screen, etc.)
  \item From the VirtualBox Device menu, select ``Insert Guest
    Additions CD Image'', and follow the prompts
  \item Once done, run \texttt{autorun.sh} from
    \texttt{/media/student/VBOXADDITIONS\_4.3.36\_105129} (or
    similar), and follow the prompts. Alternatively, if that does not
    work, try running \texttt{sudo\ bash\ VBoxLinuxAdditions.run} from
    that same directory.
  \end{itemize}

\item Reboot!
\end{itemize}


\section{Development installation}

\begin{itemize}
\item
  Installed the other packages:
  \texttt{sudo\ apt-get\ install\ clang\ emacs24\ vim\ nasm\ astyle\ tofrodos\ source-highlight\ gdb\ lldb\ doxygen\ doxygen-doc\ graphviz\ ddd\ git\ g++\ evince\ g++-multilib\ libc6-dev-i386\ libc6-dev:i386\ flex}

  % edit from the original: remove the line about other apt-get commands
  % edit from original: remove python-gpgme (and description for that), as well as the objective C packages (and associated description)
  
\item
  Ran the following two commands to change the default C/C++ compiler to
  clang:

\begin{verbatim}
sudo update-alternatives --set cc /usr/bin/clang
sudo update-alternatives --set c++ /usr/bin/clang++
\end{verbatim}
\item
  Downloaded Google Chrome from
  here\footurl{https://www.google.com/chrome/browser/desktop/index.html},
  and installed it via:

% edit from the original: replacted \texttt with \begin{verbatim} so it would lay out properly in LaTeX
  
\begin{verbatim}
sudo dpkg -i google-chrome-stable_current_i386.deb
\end{verbatim}

  \begin{itemize}
  \tightlist
  \item
    That installation did not work perfectly (which was expected), and
    to fix an installation such as this you run
    \texttt{sudo\ apt-get\ -f\ install}
  \item
    Then the .deb file was removed
  \end{itemize}
\item
  Added konsole, emacs, and chrome icons to favorites (from the K
  (start) menu, right-click and select `add to favorites'), and the task
  bar (from the favorites menu, right-click and select `add to panel';
  this may require right-clicking on the panel and selecting panel
  options -\textgreater{} panel settings prior to moving the icons)
\item
  Browser customization

  \begin{itemize}
  \tightlist

% edit from the original: changed initial tabs
  \item
    Set Chrome's home page the in-contest status page
  \item
    Chrome is set as the default browser
  \end{itemize}
\item
  I loaded up emacs from the command line, and then told it to disable
  showing the startup messages (this could also be accomplished by
  following guidelines available online
  \footurl{http://xenon.stanford.edu/~manku/dotemacs.html}.
\item
  Added a few aliases were added (the last 4 lines of .bashrc) to help
  prevent people from accidentally removing files (adding -i for rm, mv,
  and cp; and aliasing xemacs to emacs). This was done both in
  /home/student/.bashrc and /etc/skel/.bashrc.

\begin{verbatim}
alias mv='mv -i'
alias rm='rm -i'
alias cp='cp -i'
alias xemacs='emacs'
\end{verbatim}

% edit from the original: removed the following line:
%\item Cloned the github repo via \texttt{git\ clone\ https://github.com/markfloryan/pdr}; note that a  \texttt{git\ pull} will still have to be executed each time to update  it
\item
  Removed all the empty default directories in \textasciitilde{}/student
  other than Desktop and Downloads
\item
  Changed the auto-lock feature: K-menu -\textgreater{} Computer
  -\textgreater{} System Settings -\textgreater{} Desktop Behavior
  -\textgreater{} Screen Locking, uncheck the ``lock screen
  automatically\ldots{}'' button, and click Apply.
\item
  Plasmashell, which is part of the graphical window system, was
  crashing repeatedly. The reason seems to be tooltip previews of
  application windows, so these should be disabled: right click taskbar
  -\textgreater{} Task Manager Settings -\textgreater{} General
  -\textgreater{} Uncheck ``show tooltips''
\end{itemize}


\section{Programming Contest configuration
sections}\label{programming-contest-configuration-sections}


One of the sections below will detail how to turn off the
Internet for use in the actual contest, and that should only be
completed for the final contest image.

\subsection*{Programming Contest configuration}

\begin{itemize}

\item
  Install the packages:
  {\tt sudo apt-get install emacs24 vim eclipse g++ git gdb gedit openjdk-8-jdk}

  \begin{itemize}
  \tightlist
  \item
    If the Basic Installation section, above, was installed, then some
    of these packages have already been installed
  \item
    The openjdk-8-doc package is not installed here to keep the image
    size down, but the packages above install it anyway
  \end{itemize}

\item Install Eclipse
  \begin{itemize}
  \tightlist
\item  We have found that the version of Eclipse that can be installed via {\tt apt-get} is often out of date.  Thus, we installed it from the packages online\ldots
  \item Fill this in...  {\bf TODO}
\end{itemize}  

\item
  If the submission system that you are using requires installation, then download that.  We used PC$^2$, so the directions for that are included here.

  \begin{itemize}
    \tightlist
    \item Download the latest \href{http://www.ecs.csus.edu/pc2/}{PC\^{}2}
  software and unzip it in /usr/local/
  \item
    At the time of this writing, the latest version is 9.3.2-3449, and
    the direct download link is
    \href{http://www.ecs.csus.edu/pc2/code/v9/pc2-9.3.2-3449/pc2-9.3.2-3449.zip}{here};
    thus, the directory it is unzipped into would be
    \texttt{/usr/local/pc2-9.3.2}
  \item
    After unzipping, create a symlink: from /usr/local/ run (changing
    the version as necessary): \texttt{sudo\ ln\ -s\ pc2-9.3.2\ pc2}
  \item
    Edit /usr/local/pc2/pc2v9.ini and replace \texttt{localhost} on the
    \texttt{server} line (likely line 12) with the full server name of
    the primary submission server
  \end{itemize}
\item
  Create a script to restore the Internet access after the contest
  (this does not hurt anything to have it ready, as it won't do
  anything if the Internet is not turned off). This can be
  \texttt{/usr/local/bin/ restore-internet}.  As the commands therein
  need sudo, only the \texttt{student} user can run them. This is
  typically executed at the end of the contest if the contestants want
  to save their work (usually by emailing it to themselves). All that
  has to be done is to modify the default policy for the chains. Be
  sure to also run
  \texttt{chmod\ 755\ /usr/local/bin/restore-internet}.  The contents
  of the {\tt restore-internet} script are the following four lines:

\begin{verbatim}
#!/bin/bash
sudo iptables -P INPUT ACCEPT
sudo iptables -P OUTPUT ACCEPT
sudo iptables -P FORWARD ACCEPT
\end{verbatim}
\item
  Create a \texttt{/usr/local/bin/pc2team} shell script, and allow it
  to be executed via {\tt chmod 755 /usr/local/bin/pc2team}.  The
  contents of the {\tt pc2team} script file are the following four
  lines:

\begin{verbatim}
#!/bin/bash
cd
/bin/cp -f /usr/local/pc2/pc2v9.ini .
/usr/local/pc2/bin/pc2team &
\end{verbatim}
\end{itemize}

\subsection*{User account configuration}

The {\tt student} account can execute commands via {\tt sudo}, so we
have to create another account for the students to use.

\begin{itemize}
  \tightlist
\item Change the password for {\tt student} to something that they will not be able to guess.
  \item Create a {\tt hspcteam} account (or similar), and set its password
    to {\tt password} (or similar).
\item
  If there are more users in the image, then run \texttt{chmod\ 700\ /home/*} to
  prevent the second user from accessing any other user accounts.  If
  there is only one user account on the system, then this is not
  necessary.  There isn't anything of use in the {\tt student}
  account, however.
\item Log in as hspcteam; the rest of the commands in this list are from
    that account.
\item
  Add in the four aliases to .bashrc, as described above in the ``Basic
  installation'' section
\item
  Put the often used icons in the tool bar: Firefox, Emacs,
  Konsole, and Eclipse
\item
  Load up Chrome, and set the contest site as the default
  home page
\item
  Load up emacs, click on ``never show this again'', and then ``dismiss
  startup screen''; exit emacs
\item
  Remove all directories in the account other than Downloads and Desktop
\item
  Create an icon in the task bar for pc2team.  It may be easiest to
  use an icon adapted from another application.  You can find an
  appropriate logo
  online\footurl{https://www.google.com/search?q=icpc+logo} to use for that
\end{itemize}

\subsection*{Turning off Internet access (and other things)}

\begin{itemize}
\item
  Prior to the contest itself, a bunch of things need to occur to the
  image for it to work: Internet needs to be turned off, etc. These
  modifications are \textbf{\emph{NOT}} made to the image as released
  to the contestants prior to the contest, and should be made only
  right before the final image is prepared for the contest.
\item
  Edit {\tt /etc/hosts} to allow that to resolve domain names to IP
  addresses.  Specifically, the submission server(s) and the server(s)
  that have the contest materials should be included in that file, as
  follows:

\begin{verbatim}
1.2.3.4 server1.university.edu server1
5.6.7.8 server2.university.edu server2
9.10.11.12 server3.university.edu server3
\end{verbatim}
\item
  Configure iptables: this details how to turn \emph{off} network
  access. \textbf{It is ONLY to be used on the final image for a
  programming contest!!!}

  \begin{itemize}
  \item
    These directions based on the ones here\footurl{https://help.ubuntu.com/community/IptablesHowTo} and \footurl{http://stackoverflow.com/questions/21870386/iptables-block-access-to-all-ports-except-from-a-partial-ip-address}.
    We want to allow access to the submission servers, but deny
    everything else. We set the policy to all three chains to DROP, and
    add ACCEPT lines for localhost and the course servers for the INPUT
    and OUTPUT chains (we don't do anything, other than changing the
    default policy, to the FORWARD chain). For example, the following
    commands will work (assuming the servers are
    server1.cs.univeristy.edu through server3.cs.university.edu):

\begin{verbatim}
iptables -A INPUT -i lo -j ACCEPT
iptables -A OUTPUT -o lo -j ACCEPT
iptables -A INPUT -j ACCEPT -s server1.cs.university.edu
iptables -A INPUT -j ACCEPT -s server2.cs.university.edu
iptables -A INPUT -j ACCEPT -s server3.cs.university.edu
iptables -A OUTPUT -j ACCEPT -d server1.cs.university.edu
iptables -A OUTPUT -j ACCEPT -d server2.cs.university.edu
iptables -A OUTPUT -j ACCEPT -d server3.cs.university.edu
iptables -P INPUT DROP
iptables -P OUTPUT DROP
iptables -P FORWARD DROP
\end{verbatim}
  \item
    Once those commands are entered via the command line, we save the
    rules: \texttt{iptables-save\ \textgreater{}\ /etc/iptables.rules}
    (note the greater-than sign in that command)
  \item
    After saving the rules, we configure the system to apply those rules
    on boot (specifically, when bringing the network interface up). Edit
    /etc/network/interfaces. There will only be a clause for lo (the
    loopback interface) present. Under the
    \texttt{iface\ lo\ inet\ loopback} line, put in the following line
    (indented):
    \texttt{pre-up\ iptables-restore\ \textless{}\ /etc/iptables.rules}
  \end{itemize}
\item You may want to be able to reset the image between the practice contest and the full contest.  To do this, log in as {\tt student} (as you will need to use {\tt sudo}, and run the following command:
    \texttt{sudo\ /bin/cp\ -a\ /home/hspcteam\ .}; then enter the
    following script (and make it executable) as
    \texttt{/usr/local/bin/reset-hspcteam}:

\begin{verbatim}
#!/bin/bash
cd
sudo rsync -a --del hspcteam/ /home/hspcteam/
\end{verbatim}

\end{itemize}

\section{Image completion}\label{image-completion}


\subsection*{Image finalization}

\begin{itemize}
\tightlist
\item
  Reboot (rebooting also clears out /tmp)
\item
  Run \texttt{apt-get\ autoremove} and \texttt{apt-get\ clean}
\item
  Cleared browser history
\item
  To reduce the size of the hard disk (since I have to host it for
  people to download), I ran
  \texttt{sudo\ cat\ /dev/zero\ \textgreater{}\ zero;\ sudo\ /bin/rm\ -f\ zero}.
  This creates a large file of all 0's until it runs out of space, then
  removes that file.
\item
  Ran \texttt{history\ -c} to remove the history of the commands entered
  on the command line
\item
  Logged out then shut down the guest, and ran:
  \texttt{VBoxManage\ modifyhd\ disk.vdi\ -\/-compact} (using the real
  image file name instead of ``disk.vdi''). A better way would be to
  load up into recovery mode and run zerofill, but the grub menu does
  not seem to be easily available to load into recovery mode, so I
  didn't pursue it any further.
\end{itemize}

\subsection*{Increasing the disk size}

This should not be necessary, as the 20 Gb hard drive in the image has
about half of that space available.  Should you want to do this, you
can find the directions online
\footurl{http://aaronbloomfield.github.io/slp/docs/virtualbox-image-details.html}.
