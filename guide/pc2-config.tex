As of the 2012 contest (i.e., March 17, 2012), the latest version of
PC$^2$ was 9.2.1.  There are other contest systems, including KATTIS
(which is used at many world finals).  However, PC$^2$ is used at the
regional contests~-- and typically used for local practices~-- so that is what
we opted to use.

This section assumes that the reader is familiar with PC$^2$.

An ideal way to configure a contest is to launch a server through an
ssh shell, and have all the other PC$^2$ clients connect to that
server.

\subsection{Configuration files}

There will be two different versions of the configuration files: one
on the remote server, and one that all the clients use.  Here is the
client configuration file:

\begin{verbatim}
[client]
server=hostname.domain.edu:50002
minSecsBetweenUpdate=0
[server]
\end{verbatim}

There are many other options in the configuration file~-- such as
having multiple sites~-- but we did not need them.  The {\tt server}
line specifies the server and port to connect to.  For the
configuration file on the remote server, just change the hostname from
{\tt hostname.domain.edu} to {\tt localhost}.  The {\tt
minSecsBetweenUpdate} line forces the various clients (such as the
scoreboard) to immediately write the updated scoreboard files to the
disk, rather than buffering them for a period of time.

\subsection{Pre-configuring PC$^2$}

PC$^2$ can be pre-configured, and the configuration files moved over
to the server that is actually running the contest.  This makes it
easy to configure it locally, and then upload the files to the remote
server.

When you run pc2server (see next section), there are two directories
created (logs/ and profiles/) and one file (profiles.properties).  The
logs directory need not be preserved, unless you want to maintain a
log of your PC$^2$ configuration.

In the profiles/ directory, there will be a directory with a random
name (such as 'P3b9dcd2a-1d67-4106-8e2b-6b407198e16b/').  And in there
are a number of files and directories:

\begin{itemlist}
\item db.1/
\item logs/
\item packets/
\item pc2.startup-0.log
\item pc2.startup-0.log.lck
\end{itemlist}

One the PC$^2$ configuration is complete (and pc2server is shut
down!), only the files in the db.1/ directory actually need to be
preserved.  Thus, a fully configured contest would have these
files/directories:

\begin{itemlist}
\item ./profiles/
\item ./profiles/P3b9dcd2a-1d67-4106-8e2b-6b407198e16b/
\item ./profiles/P3b9dcd2a-1d67-4106-8e2b-6b407198e16b/db.1/
\item ./profiles/P3b9dcd2a-1d67-4106-8e2b-6b407198e16b/db.1/settings.dat
\item ./profiles/P3b9dcd2a-1d67-4106-8e2b-6b407198e16b/db.1/recovery.key
\item ./profiles/P3b9dcd2a-1d67-4106-8e2b-6b407198e16b/db.1/contest.key
\item ./pc2v9.ini
\item ./profiles.properties
\end{itemlist}

The pc2v9.ini file should have {\tt localhost} as the hostname, as
that is the directory that you will be running pc2server out of.  Note
that you may want to include a .htaccess file to prevent anybody from
viewing the contents of this directory, if it is going to be on a web
server (it only needs two lines: 'Order deny,allow', and 'deny from
all').

PC$^2$ version 9.3 promises to include more detailed profiles, which
would allow for easy switching between the practice contest
configuration and the actual contest configuration.  While PC$^2$
version 9.3 was released, it was so buggy that it was retracted, and
9.2.1 is the current latest version (as of March 17, 2012).  Thus, you
will probably want two different directories with the above setup: one
called {\tt contest/} and one called {\tt practice/}.  This makes it
easy to switch between the two contests: just kill the server, cd into
the other directory, and re-launch the server.

While most of the files are fairly small, the size of settings.dat
will be proportional to the size of the files that you configure
PC$^2$ with: judging input, judging output, validators, etc.  Thus, if
you have a 10 Mb output file for the judging data, the settings.dat
will be slightly larger than 10 Mb.

Note that settings.dat is a binary file (it's Java Serialized data),
so it will not work well with revision control software, such as SVN.
And if you make a small change (such as one account name), SVN will
copy the entire updated file into the repository; too many of these
minor updates will dramatically increase the size of the repository.
This may not be a problem, but it may also be, depending on if you are
using revision control and where you are hosting it.

\subsection{pc2server}

Run {\tt pc2server}.  The initial login username is site1, and the
password is also site1, and you can pick anything for the contest
password (but remember it!).  There is very little configuration
needed to be done from the pc2server GUI~-- the only one is in the
Sites tab, where you can change the Site Title (to 'UVa' instead of
'Site 1') and the password (that replaces 'site1', above).  All the
other configuration can be done from pc2admin.

When running it remotely, here is a command to kick off the server
(this is all on one line):

\begin{verbatim}
nohup pc2-9.2.4/bin/pc2server --nogui --login site1 --password site1
      --contestpassword mypassword | tee server.out &
\end{verbatim}

The {\tt nohup} (for 'no hang-up') prevents terminating the server
upon a logout, and the 'tee' causes output to be displayed both to
standard output and to the file specified.  Note that the {\tt
--password} password starts as 'site1', but can be changed via the
GUI.  The contest password is what you set as the contest password on
the first login (i.e., the running of this command).  The {\tt
--nogui} flag prevents loading of the GUI interface.

Note that closing the server (through the 'Exit' button in the GUI
(admin GUI?) or killing the process) will terminate all connected
clients.

\subsection{pc2admin}

\subsubsection{Configuring accounts}

One can easily create accounts in pc2admin via the 'generate' button.
And while one can set all the options through the pc2admin interface,
it is much easier to load all of them into a CSV file (tab separated),
and load them into PC$^2$ directly.

The {\tt accounts.csv} file needs to contain the following fields:

\begin{itemlist}
\item {\tt site} is the site number that the account is going to be
used on; if a single-site contest (as most high school contests are),
then they are all '1'.
\item {\tt account} is the account name: 'administrator1', 'team1',
etc.  Numbering should start at 1 for all accounts to enable easy
loading of the file into PC$^2$.  This is the login name.
\item {\tt displayname} is the name that should be displayed.  For
contest staff, this is their full name.  For teams, this is their
school name and team name.
\item {\tt password} is the login password; see below for how to
easily generate random passwords.
\item {\tt group} is if they are in a group (i.e., all teams from one
high school are in a given group).  We have not needed this feature.
\item {\tt permdisplay} is whether they have permission to be
displayed on the scoreboard.  This should be 'true' for all teams that
are competing, and 'false' for any other team accounts (extra team
accounts, a team account for staff testing, etc.).  If another team
shows up, one can always toggle this permission in the pc2admin GUI.
\item {\tt permlogin} is whether they can log in.  Should be 'false'
for any extra accounts that are not being used, and 'true' otherwise;
this can also be toggled through the pc2admin GUI.
\item {\tt externalid} is not something we used, and was left blank.
\item {\tt alias} is not something we used, and was left blank.
\item {\tt permpassword} is whether they have permission to change
their password.  Set to 'true' or 'false' as you would prefer.
\end{itemlist}

In LibreOffice (and, presumably, in Microsoft Excel), one can generate
a random 10-digit password via the following command: 

\begin{verbatim}
=concatenate(char(randbetween(97,122)),char(randbetween(97,122)),
char(randbetween(97,122)),char(randbetween(97,122)),char(randbetween(97,122)),
char(randbetween(97,122)),char(randbetween(97,122)),char(randbetween(97,122)),
char(randbetween(97,122)),char(randbetween(97,122)))
\end{verbatim}

Note that when the file is saved as a CSV, the actual value of the
password is saved, not the formula.

To load the accounts.csv file, you first have to generate the same
number of accounts via the pc2admin GUI (Configure Contest -$>$
Accounts -$>$ Generate).  If not all the accounts have been generated,
then pc2admin will not load the accounts.csv file.  If you are
changing properties that are not account, displayname, or password
(i.e., if you are changing any of the permissions), you have to click
the 'Include unchanged accounts' check box for those changes to take
effect.

As the accounts are likely to change often, and it's a text file, it
works much better with revision control to edit this file, and always
load the accounts when launching the practices or system tests; this
prevents the settings.dat file from having any modifications.

\subsubsection{Configuring problems}

(including setting up the auto-judge)

Configuring a custom validator is described in the end of
section~\ref{CustomValidators}.

\subsubsection{Other configuration details}

\begin{itemlist}
\item setting the time for the contest
\item languages that they can use
\item \ldots
\end{itemlist}

\subsection{pc2team}

There isn't much to talk about this~-- once the accounts are set up,
and the passwords are distributed, teams can login.  Note that if they
can access pc2team before the contest clock starts, they will be able
to see the names of the problems.

Talk about the 'Test' button\ldots (Ben?)


\subsection{pc2judge}

There will need to be many judges.  Each problem will probably need an
auto-judge, which is a separate login.  And the human judges will each
need an account.  It is desirable to have the auto-judges running on a
separate computer, as every time a submission comes in, it pops up
many dialog boxes (and if you happen to hit the space bar, because you
were typing something, it may click the 'cancel' button on those
dialog boxes).

We found that running 10 auto-judges on a single computer caused too
many problems: memory and CPU usage.  We would have 2 problems
auto-judged on a single computer, and typically tried to balance the
problems (so the hardest problem was auto-judged on the same computer
as the easiest problem, etc.).

With the configuration set above, the auto-judge will determine the
right answer, and then a human judge to review and release.

\subsubsection{Judging problems}

PC$^2$ cannot distinguish between a wrong answer (WA) and a run-time
error (RTE).  In the case of the built-in validator, they will all
report as WA.  If a custom validator is being used (see Validators,
below), then the will all report as ``No~-- Undetermined''.  In both
cases, when the human pulls up the output, a RTE will show the
non-zero exit code in big, bold, red letters at the bottom; a WA will
not show this.  So while it's easy for a human to determine which is
which, the auto-judges cannot.

\subsubsection{Using pc2judge}

Depending upon the number of judges, it may be most useful to take
advantage of the filter features. Each judge is responsible for
dealing with their own subset of problems, which helps with
consistency and keeps things organized during a rush of problem
submissions and/or clarifications. In practice, all of the judges will
be collaborating, but this can help things running smoothly. Many
problems involving a validator will come in as "No~--Undetermined",
and verifying the input will be the only way to get an actual response
is to check the output manually. Runtime errors are easy to identify,
as the error code will be displayed at the bottom of the
window. "Wrong Answer" will be the most likely response needed,
although "Output Format" will also arise. "Output Format" is rather
difficult to judge, unless you can convince yourself that everything
is correct but in the wrong place. It is generally advised to bias in
favor of a "Wrong Answer." Further, the juding system does not allow
"Time Limit Exceeded" to be simply accepted as a correct judging, and
so this will need to be handled manually.

\subsubsection{Balloon distribution}
\label{sec:pc2judge-balloons}

Balloon people need to prepare beforehand with pc2 so that they know
how to set it up and deal with the system in order to get balloons
delivered on time.

\subsection{pc2board}

